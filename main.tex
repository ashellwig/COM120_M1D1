\documentclass[stu,12pt]{apa7}
  \usepackage{times}               % Times New Roman Font Face
  \usepackage[american]{babel}      % Localization
  \usepackage[utf8]{inputenc}      % Input Encoding
  \usepackage{hyperref}            % Hyperlinks
  \usepackage{enumitem}            % Additional Enumeration Environment Settings
  \usepackage{geometry}            % Page Layout
  \usepackage{soul}                % Text Highlighting
  \usepackage{graphicx}            % Images
  \usepackage{csquotes}            % Quoting Environment
  \usepackage{bookmark}            % Required by `csquotes'
  \usepackage{mdframed}            % Colorful Tex-Box Environment
  \usepackage[toc]{appendix}       % Appendix
  \usepackage{fancyhdr}            % Headings and Footers
  \usepackage[%
    style=apa,%
    sortcites=true,%
    sorting=nyt%
  ]{biblatex}
  \usepackage{xcolor}

  % Bibliography Setup
  %% Language Mappings
  \DeclareLanguageMapping{english}{english-apa}
  \DeclareLanguageMapping{american}{american-apa}
  %% Bibliography File Path
  \addbibresource{main.bib}
  %% Categories for Specified Bibliography Items
  %%% Category for sources not referenced in-text
  \DeclareBibliographyCategory{consulted}
  \addtocategory{consulted}{noauthor_primer_2012}
  \addtocategory{consulted}{kondo_developing_2020}

  % Hyperlink Setup
  \hypersetup{
    colorlinks = true,
    urlcolor = blue,
    linkcolor = blue,
    citecolor = blue
  }

  % Page and Text Layout
  \geometry{%
    a4paper,%
    top=1in,%
    bottom=1in,%
    left=1in,%
    right=1in%
  }
  \setlength{\headheight}{15pt}

  % Header
  \lhead{COM120CG1-M1D1}

  % Title Page
  \title{%
    M1D1: What Makes an Ideal Interpersonal Communicator
  }
  \shorttitle{Module 1 Discussion 1}
  \author{Ashton Hellwig}
  \authorsaffiliations{Department of Mathematics, Front Range Community College}
  \course{COM115: Interpersonal Communication}
  \professor{Richard Thomas}
  \duedate{November 07, 2020 23:59:59 MDT}
  \date{\today}
  \abstract{%
    \textbf{Overview}\\%
    Have you considered what makes an ideal interpersonal communicator? This is
      different than what makes a great public speaker! Remember, interpersonal
      communication is one-on-one communication rather than communication that
      reaches a broader group.\\%

    Before you can evaluate your own interpersonal abilities, it’s essential
      that you identify good communicators and their skills.\\%

    In the movie clip appearing within the topic titled, ``Self-Concept and
      Perception'', which is accessible within the ``Read/View'' topic of Module
      1 Content, the actor portraying President Roosevelt demonstrates several
      effective interpersonal communication skills. For example, while speaking
      to the actor portraying King George, he demonstrates his own difficulties
      with how people perceive him by moving around the room without the aid of
      his wheelchair.\\%

    What do you think makes an ideal interpersonal communicator? In this
      activity, we will explore this question in depth.\\%

    You should spend approximately 4 hours on this assignment.
  }

\begin{document}
  % Title Page
  \maketitle

  % Initial Post
  \section{Initial Post}
    \subsection*{Instructions}
      \begin{enumerate}
        \item In addition to your research, think of two people who you believe
          to be great interpersonal communicators: one in your personal life (a
          family member, friend, or loved one) and one in your professional life
          (a work or church colleague, professor, or customer). Also, think
          about the other skills the actor portraying President Roosevelt
          demonstrated in the movie clip.
        \item Create a list of effective interpersonal communication skills, and
          post them on this discussion board. Include both skills you found in
          your readings and research and skills you have seen displayed by good
          communicators. \textcolor{green}{Include five (5) interpersonal
          communication skills with your list. Include a minimum of
          \(2\rightarrow 3\) sentences of explanation about each interpersonal
          communication skill and why the skill is important}.
        \item \textcolor{green}{At the end of your Main Post, add a minimum
          of three resources which are correctly formatted in accordance with
          either APA Style or MLA Style (your choice)}.
        \item Read and Comment on the Main post submitted by at least two
          classmates who identified one or more interpersonal communication
          skills which are either the same as or different from yours. Did their
          research result in a change to your list? Why or why not?
      \end{enumerate}

    \newpage
    % What Makes an Ideal Interpersonal Communicator
    \paragraph{Qualities of an Ideal Interpersonal Communicator}
      I believe that the ideal ``interpersonal communicator'', if such a being
        exists, has qualities that make someone who one would \textit{want}
        to go to for advice. Many times, people are afraid of asking for help
        or advice. More often than not, I feel most people are generally afraid
        of even describing experience or a situation to others for fear of
        judgement or someone coming out of the conversation feeling less
        intelligent or ``smaller'' as a person. The ideal communicator is one
        who someone can express what they are really thinking and believe with
        one another in a civilized manner and come out of the conversation
        with more knowledge than they had when beginning the interaction.

    % Two Examples of Good Communicators in Your Life
    \paragraph{Think of two people you consider to be good interpersonal
      communicators and explain why each one was chosen}
      It is actually suprisingly difficult for me to think of one, let alone
        two examples of people I believe to be even half-way decent at the
        ``art'' of communication. Most everyone I know has not yet mastered
        this life-skill. In the video, it is clear listening is the priority
        for President Roosevelt throughout the conversation. He does, however,
        break some of the ``Rules'' described by Celeste Headlee in the Ted Talk
        shown in this module's exploration component. In that video, she
        described the act of expressing your pain or relating your own situation
        to theirs. I find that \emph{most, if not all} people (including myself)
        do this. After the King's stuttering, the President showed how confident
        he is with his disability and how little it matters in how the public's
        \hl{perception} of you changes. In some instances this could be seen as
        comforting, but in others could be seen as a rude comparison or as
        putting off a more ``look at me'' attitude.

    % Detailed List of Interpersonal Communication Skills
    \paragraph{Detailed List of Interpersonal Communication Skills}
      There are many skills utilized by effective communicators depending on the
        nature of their interaction with the other individuals in question.
      \begin{enumerate}
        \item \textit{A Strong Hold on Psychological Context}
          \begin{itemize}
            \item \hl{Psychological Context} is maybe not so much an
              interpersonal communication ``skill'', but rather it is an
              important concept for one to grasp in order to effectively
              communicate any message. Interpersonal communication deals with
              incredibly personal relationships and as such the environment's
              context needs to be taken into account in order to not prevent
              the delivery of your message due to the other party being unnerved
              or too stressed out to effectively understand what you are
              trying to convey \parencite{noauthor_communication_2013}.
              Because interpersonal communication concerns all intimate and
              personal relationships, it is important that the speakers show an
              understanding of how one-another communicates and adjusts how they
              speak to best fit the situation.
          \end{itemize}
        \item \textit{%
          Enter Every Conversation with the intention to understand, rather %
            than with the intention to reply%
        }
          \begin{itemize}
            \item This skill is important and was described to us in Module
              1's ``\textit{exploration}'' content. The reason this is important
              is because, as noted in the next item, \emph{listening} is
              \textbf{the} most important skill within the realm of
              interpersonal communication \parencite{headlee_10_2015}. If
              one person is already conjuring their retort to what the other
              individual is saying, at that point they have stopped listening.
              If you are not listening, you cannot understand.
          \end{itemize}
        \item Effectively Utilize Silence
          \begin{itemize}
            \item Silence can be situational in its ``\textit{appropriatness}''
              within a conversation. In some situations, it can help bond the
              two individuals who are speaking whereas in others it can
              drive eachother away
              \parencite[pp. 600]{gerd_antos_handbook_2008}. When appropriate,
              silence can even be an intimate and comfortable experience, rather
              than both a repulsive and anxiety-inducing one.
          \end{itemize}
      \end{enumerate}

  % Replies
  % %! TEX root='..\main.tex'

\part{}


  % Bibliography
  %% Works Cited
  \newpage
  \printbibliography[%
    title={Works Cited},%
    heading={bibintoc},%
    notcategory={consulted}%
  ]

  %% Works Consulted
  \newpage
  \nocite{*}
  \printbibliography[%
    title={Works Consulted},%
    heading={bibintoc},%
    category={consulted}%
  ]
\end{document}

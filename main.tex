\documentclass[stu,12pt]{apa7}
  \usepackage{times}
  \usepackage[english]{babel}
  \usepackage[utf8]{inputenc}
  \usepackage{xcolor}
  \usepackage{hyperref}
  \usepackage{enumitem}
  \usepackage{geometry}
  \usepackage{soul}
  \usepackage{graphicx}
  \usepackage{csquotes}
  \usepackage{bookmark}
  \usepackage{mdframed}
  \usepackage[toc]{appendix}
  % \usepackage[backend=biber,style=apa]{biblatex}
  \usepackage{fancyhdr}
  \usepackage{mdframed}

  % Bibliography Setup
  % \addbibresource{main.bib}

  % Hyperlink Setup
  \hypersetup{
    colorlinks = true,
    urlcolor = blue,
    linkcolor = blue,
    citecolor = blue
  }

  % Page and Text Layout
  \geometry{%
    a4paper,%
    top=1in,%
    bottom=1in,%
    left=1in,%
    right=1in%
  }
  \setlength{\headheight}{15pt}

  % Header
  \lhead{COM120CG1-M1D1}

  % Title Page
  \title{%
    M1D1: What Makes an Ideal Interpersonal Communicator
  }
  \shorttitle{Module 1 Discussion 1}
  \author{Ashton Hellwig}
  \authorsaffiliations{Department of Mathematics, Front Range Community College}
  \course{COM115: Interpersonal Communication}
  \professor{Richard Thomas}
  \duedate{November 07, 2020 23:59:59 MDT}
  \date{\today}
  \abstract{%
    \textbf{Overview}\\%
    Have you considered what makes an ideal interpersonal communicator? This is
      different than what makes a great public speaker! Remember, interpersonal
      communication is one-on-one communication rather than communication that
      reaches a broader group.\\%

    Before you can evaluate your own interpersonal abilities, it’s essential
      that you identify good communicators and their skills.\\%

    In the movie clip appearing within the topic titled, ``Self-Concept and
      Perception'', which is accessible within the ``Read/View'' topic of Module
      1 Content, the actor portraying President Roosevelt demonstrates several
      effective interpersonal communication skills. For example, while speaking
      to the actor portraying King George, he demonstrates his own difficulties
      with how people perceive him by moving around the room without the aid of
      his wheelchair.\\%

    What do you think makes an ideal interpersonal communicator? In this
      activity, we will explore this question in depth.\\%

    You should spend approximately 4 hours on this assignment.
  }

\begin{document}
  % Title Page
  \maketitle

  % Initial Post
  \section{Initial Post}
    \subsection*{Instructions}
      \begin{enumerate}
        \item In addition to your research, think of two people who you believe
          to be great interpersonal communicators: one in your personal life (a
          family member, friend, or loved one) and one in your professional life
          (a work or church colleague, professor, or customer). Also, think
          about the other skills the actor portraying President Roosevelt
          demonstrated in the movie clip.
        \item Create a list of effective interpersonal communication skills, and
          post them on this discussion board. Include both skills you found in
          your readings and research and skills you have seen displayed by good
          communicators. \textcolor{green}{Include five (5) interpersonal
          communication skills with your list. Include a minimum of
          \(2\rightarrow 3\) sentences of explanation about each interpersonal
          communication skill and why the skill is important}.
        \item \textcolor{green}{At the end of your Main Post, add a minimum of
          three resources which are correctly formatted in accordance with
          either APA Style or MLA Style (your choice)}.
        \item Read and Comment on the Main post submitted by at least two
          classmates who identified one or more interpersonal communication
          skills which are either the same as or different from yours. Did their
          research result in a change to your list? Why or why not?
      \end{enumerate}

    \newpage
    \subsection{Introduction}
      This is just a simple little placeholder.

  % Replies
  % %! TEX root='..\main.tex'

\part{}


  % Bibliography
\end{document}

\documentclass[stu,12pt]{apa7}
  \usepackage{times}               % Times New Roman Font Face
  \usepackage[american]{babel}      % Localization
  \usepackage[utf8]{inputenc}      % Input Encoding
  \usepackage{hyperref}            % Hyperlinks
  \usepackage{enumitem}            % Additional Enumeration Environment Settings
  \usepackage{geometry}            % Page Layout
  \usepackage{soul}                % Text Highlighting
  \usepackage{graphicx}            % Images
  \usepackage{csquotes}            % Quoting Environment
  \usepackage{bookmark}            % Required by `csquotes'
  \usepackage{mdframed}            % Colorful Tex-Box Environment
  \usepackage[toc]{appendix}       % Appendix
  \usepackage{fancyhdr}            % Headings and Footers
  \usepackage[%
    style=apa,%
    sortcites=true,%
    sorting=nyt%
  ]{biblatex}
  \usepackage{xcolor}

  % Bibliography Setup
  %% Language Mappings
  \DeclareLanguageMapping{english}{english-apa}
  \DeclareLanguageMapping{american}{american-apa}
  %% Bibliography File Path
  \addbibresource{main.bib}
  %% Categories for Specified Bibliography Items
  %%% Category for sources not referenced in-text
  \DeclareBibliographyCategory{consulted}
  \addtocategory{consulted}{noauthor_primer_2012}
  \addtocategory{consulted}{kondo_developing_2020}

  % Hyperlink Setup
  \hypersetup{
    colorlinks = true,
    urlcolor = blue,
    linkcolor = blue,
    citecolor = blue
  }

  % Page and Text Layout
  \geometry{%
    a4paper,%
    top=1in,%
    bottom=1in,%
    left=1in,%
    right=1in%
  }
  \setlength{\headheight}{15pt}

  % Header
  \lhead{COM120CG1-M1D1}

  % Title Page
  \title{%
    M1D1: What Makes an Ideal Interpersonal Communicator
  }
  \shorttitle{Module 1 Discussion 1}
  \author{Ashton Hellwig}
  \authorsaffiliations{Department of Mathematics, Front Range Community College}
  \course{COM115: Interpersonal Communication}
  \professor{Richard Thomas}
  \duedate{November 07, 2020 23:59:59 MDT}
  \date{\today}
  \abstract{%
    \textbf{Overview}\\%
    Have you considered what makes an ideal interpersonal communicator? This is
      different than what makes a great public speaker! Remember, interpersonal
      communication is one-on-one communication rather than communication that
      reaches a broader group.\\%

    Before you can evaluate your own interpersonal abilities, it’s essential
      that you identify good communicators and their skills.\\%

    In the movie clip appearing within the topic titled, ``Self-Concept and
      Perception'', which is accessible within the ``Read/View'' topic of Module
      1 Content, the actor portraying President Roosevelt demonstrates several
      effective interpersonal communication skills. For example, while speaking
      to the actor portraying King George, he demonstrates his own difficulties
      with how people perceive him by moving around the room without the aid of
      his wheelchair.\\%

    What do you think makes an ideal interpersonal communicator? In this
      activity, we will explore this question in depth.\\%

    You should spend approximately 4 hours on this assignment.
  }

\begin{document}
  % Title Page
  \maketitle

  % Initial Post
  \section{Initial Post}
    \subsection*{Instructions}
      \begin{enumerate}
        \item In addition to your research, think of two people who you believe
          to be great interpersonal communicators: one in your personal life (a
          family member, friend, or loved one) and one in your professional life
          (a work or church colleague, professor, or customer). Also, think
          about the other skills the actor portraying President Roosevelt
          demonstrated in the movie clip.
        \item Create a list of effective interpersonal communication skills, and
          post them on this discussion board. Include both skills you found in
          your readings and research and skills you have seen displayed by good
          communicators. \textcolor{green}{Include five (5) interpersonal
          communication skills with your list. Include a minimum of
          \(2\rightarrow 3\) sentences of explanation about each interpersonal
          communication skill and why the skill is important}.
        \item \textcolor{green}{At the end of your Main Post, add a minimum
          of three resources which are correctly formatted in accordance with
          either APA Style or MLA Style (your choice)}.
        \item Read and Comment on the Main post submitted by at least two
          classmates who identified one or more interpersonal communication
          skills which are either the same as or different from yours. Did their
          research result in a change to your list? Why or why not?
      \end{enumerate}

    \newpage
    % What Makes an Ideal Interpersonal Communicator
    \paragraph{Qualities of an Ideal Interpersonal Communicator}
      I believe that the ideal ``interpersonal communicator'', if such a being
        exists, has qualities that make someone who one would \textit{want}
        to go to for advice. Many times, people are afraid of asking for help
        or advice. More often than not, I feel most people are generally afraid
        of even describing experience or a situation to others for fear of
        judgement or someone coming out of the conversation feeling less
        intelligent or ``smaller'' as a person. The ideal communicator is one
        who someone can express what they are really thinking and believe with
        one another in a civilized manner and come out of the conversation
        with more knowledge than they had when beginning the interaction.

    % Two Examples of Good Communicators in Your Life
    \paragraph{Think of two people you consider to be good interpersonal
      communicators and explain why each one was chosen}
      It is actually surprisingly difficult for me to think of one, let alone
        two examples of people I believe to be even half-way decent at the
        ``art'' of communication. Most everyone I know has not yet mastered
        this life-skill. In the video, it is clear listening is the priority
        for President Roosevelt throughout the conversation. He does, however,
        break some of the ``Rules'' described by Celeste Headlee in the Ted Talk
        shown in this module's exploration component. In that video, she
        described the act of expressing your pain or relating your own situation
        to theirs. I find that \emph{most, if not all} people (including myself)
        do this. After the King's stuttering, the President showed how confident
        he is with his disability and how little it matters in how the public's
        \hl{perception} of you changes. In some instances this could be seen as
        comforting, but in others could be seen as a rude comparison or as
        putting off a more ``look at me'' attitude.

    % Detailed List of Interpersonal Communication Skills
    \paragraph{Detailed List of Interpersonal Communication Skills}
      There are many skills utilized by effective communicators depending on the
        nature of their interaction with the other individuals in question.
      \begin{enumerate}
        \item \textit{A Strong Hold on Psychological Context}
          \begin{itemize}
            \item \hl{Psychological Context} is maybe not so much an
              interpersonal communication ``skill'', but rather it is an
              important concept for one to grasp in order to effectively
              communicate any message. Interpersonal communication deals with
              incredibly personal relationships and as such the environment's
              context needs to be taken into account in order to not prevent
              the delivery of your message due to the other party being unnerved
              or too stressed out to effectively understand what you are
              trying to convey \parencite{noauthor_communication_2013}.
              Because interpersonal communication concerns all intimate and
              personal relationships, it is important that the speakers show an
              understanding of how one-another communicates and adjusts how they
              speak to best fit the situation.
          \end{itemize}
        \item \textit{%
          Enter Every Conversation with the intention to understand, rather %
            than with the intention to reply%
        }
          \begin{itemize}
            \item This skill is important and was described to us in Module
              1's ``\textit{exploration}'' content. The reason this is important
              is because, as noted in the next item, \emph{listening} is
              \textbf{the} most important skill within the realm of
              interpersonal communication \parencite{headlee_10_2015}. If
              one person is already conjuring their retort to what the other
              individual is saying, at that point they have stopped listening.
              If you are not listening, you cannot understand.
          \end{itemize}
        \item Effectively Utilize Silence
          \begin{itemize}
            \item Silence can be situational in its ``\textit{appropriatness}''
              within a conversation. In some situations, it can help bond the
              two individuals who are speaking whereas in others it can
              drive each other away
              \parencite[pp. 600]{gerd_antos_handbook_2008}. When appropriate,
              silence can even be an intimate and comfortable experience, rather
              than both a repulsive and anxiety-inducing one.
          \end{itemize}
      \end{enumerate}

  % Replies
  %! TEX root=../main.tex

\section{Responses}
  \subsection{Response 1}
    \begin{quotation}
      ``An understanding is the best thing in the world'' is what my dad would
        say often (Hart, 1980). He meant for you to be clear with your words so
        that others understand you. Even with him saying that to my seven
        siblings and me, I have not always used his advice, but I know this is
        the beginning to an ideal interpersonal communicator. A coworker, Carla,
        has a great way with requesting things. She is clear, direct, and asks
        for feedback as suggested by a good communicator (Stobierski, 2019). As
        an example, she may ask me if I would go get a catheter, and if I know
        where it is. The question forces me to answer yes or no to retrieving
        the catheter, and to answer if I know where it is. She is also an
        attentive listener. I’m usually always walking and doing four things at
        once or multitasking, but, when I ask her for a something, she stops,
        faces me, and acknowledges me (Headlee, 2016). Carla also uses a calm
        tone of voice. She gets loud when she laughs or when conversations are
        funny, but Carla has a calm tone. She is in control and not much can go
        array without her handling it. My brother, Jerry, uses particular body
        language when he talks to me, and I feel he is really listening to me
        when he does this. For instance, he stands straight. His hands are
        either to his side or behind him, and he faces you. His eyes are
        slightly squinted as if he is focused only on you. Both Carla and Jerry
        used an interaction model of communication where we had to talk to each
        other, verbally and non-verbally, about a request (Jones, 2013).
    \end{quotation}

    \paragraph{This is a response to Diata Hart on Post ID 43184821}
      I really enjoy the quote from your father, as it illustrates a problem
        many have in terms of communicating with each other. I find that even if
        one may \emph{not} be understood when speaking to someone, the other
        party may not make it very clear if that is the case. Not many people
        enjoy admitting to the fact they we may not know \textit{everything}.

      I, too, am envious that you were able to pick someone out of your
        workplace environment to illustrate the personification of an ideal
        interpersonal communicator. It tends to be rare that most of us get
        along with coworkers (in the working environment, not personally).
        Individuals whom are able to accept feedback and criticism with grace
        are a rarity in today's world.

  %! TEX root=../main.tex

\subsection{Response 2}
  \begin{quotation}
    One person in my life that is very good at interpersonal communication is my
      dad. He works as a hotel broker and also has his own nonprofit business.
      He is a business man so interpersonal communication is a huge part of his
      job. He's constantly on the phone talking to customers and partners in
      the hopes of making deals. I've seen over the years that most of his
      customers end up becoming close friends of his. This is part of why I see
      him as such a good interpersonal communicator. He works hard to get to
      know everyone he speaks to whether it's for his work or just in life.
      Every time I go out in public with him, I make the joke that he has more
      friends than me because he's able to have personal conversations with
      everyone he sees. Part of what makes my dad such a good communicator is
      that he is extremely kind. This makes him easy to talk to and more
      trustworthy allowing others to open up to him. A person in my professional
      life who is a good interpersonal communicator is my boss. She's been very
      successful at forming relationships with everyone who works there that
      shows she's there for us and cares, but is also clear that she is our
      boss, not our friend. This is important because we are able to easily
      communicate with her whenever we have problems or concerns about something
      while also professionally speaking to her about our position.

    There are many things that make a good interpersonal communicator. Some
      important factors are empathy, listening, clarity in speech, managing
      emotions, and nonverbal communication. The journal, Effective
      Interpersonal Communication: A Practical Guide to Improve Your Life,
      talks about the importance of interpersonal communication within nurses
      and their patients. They explain that good interpersonal communication
      reduces stress, promotes wellness, and overall improves their patients
      quality of life. One of the key variables they say makes a good
      interpersonal communicator is empathy. Empathy is extremely important
      because it helps someone to understand another and relate to their
      emotions. Understanding emotions is so important in communication because
      otherwise there are no connections made. Empathy is why my dad is such a
      good communicator. Another important factor in good interpersonal
      communication is actually silence, and within silence, listening. The
      journal, Human Communication and Effective Interpersonal Relationships: An
      Analysis of Client Counseling and Emotional Stability, talks about
      interpersonal communication within counselors and their patients. They
      explain that silence is actually a very important part of communication
      because the act of not speaking and listening to the person they are
      communicating shows that they find what the other person is saying is
      important to them. One important part of this is to actually listen and
      the facial expressions that they show is important to show this. Without
      listening to the other person, communication doesn't go very far. Although
      both these articles are talking about nurses and therapists with clients,
      they are still essential with daily interpersonal communications. Another
      important factor is clarity in speech. If you aren't clear about what
      you're communicating it won't go very far. Without getting the point
      across, there's not a huge point in communication. Without clear points,
      it tends to be more of a ramble or a rant. Sometimes people aren't able
      to speak clearly because their emotions are too overwhelming. It's
      important in communication to manage emotions in order to have an
      effective conversation. This is very common when people are struggling
      with anger or sadness. Sometimes their emotions make it too difficult to
      effectively communicate. Finally, nonverbal communication is very
      important in interpersonal communication. Things like facial expressions,
      posture, and gesture show your attitude towards the conversation. The
      article, Listening and Interpersonal Skills Review, explains these
      nonverbal communications and how they are apart of communication. For
      example, they explain that gestures can replace words by signaling whether
      the conversation has come to an end or not. These can show whether or not
      you're actually paying attention or if you show any interest in what they
      other person is communicating. Nonverbal communication can make or break a
      conversation. It's very important to pay attention to these because
      sometimes nonverbal communication can happen unintentionally, but they
      make a huge difference in how the other person views your position in the
      conversation. These are all factors that both my dad and boss use in order
      to properly communicate. Without using these skills, interpersonal
      communication isn't very effective.
  \end{quotation}

  \paragraph{This is a response to Julayne Kilcullen on Post ID 43148692}
    Placeholder.


  % Bibliography
  %% Works Cited
  \newpage
  \printbibliography[%
    title={Works Cited},%
    heading={bibintoc},%
    notcategory={consulted}%
  ]

  %% Works Consulted
  \newpage
  \nocite{*}
  \printbibliography[%
    title={Works Consulted},%
    heading={bibintoc},%
    category={consulted}%
  ]
\end{document}

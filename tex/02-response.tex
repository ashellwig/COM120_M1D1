%! TEX root=../main.tex

\subsection{Response 2}
  \begin{quotation}
    One person in my life that is very good at interpersonal communication is my
      dad. He works as a hotel broker and also has his own nonprofit business.
      He is a business man so interpersonal communication is a huge part of his
      job. He's constantly on the phone talking to customers and partners in
      the hopes of making deals. I've seen over the years that most of his
      customers end up becoming close friends of his. This is part of why I see
      him as such a good interpersonal communicator. He works hard to get to
      know everyone he speaks to whether it's for his work or just in life.
      Every time I go out in public with him, I make the joke that he has more
      friends than me because he's able to have personal conversations with
      everyone he sees. Part of what makes my dad such a good communicator is
      that he is extremely kind. This makes him easy to talk to and more
      trustworthy allowing others to open up to him. A person in my professional
      life who is a good interpersonal communicator is my boss. She's been very
      successful at forming relationships with everyone who works there that
      shows she's there for us and cares, but is also clear that she is our
      boss, not our friend. This is important because we are able to easily
      communicate with her whenever we have problems or concerns about something
      while also professionally speaking to her about our position.

    There are many things that make a good interpersonal communicator. Some
      important factors are empathy, listening, clarity in speech, managing
      emotions, and nonverbal communication. The journal, Effective
      Interpersonal Communication: A Practical Guide to Improve Your Life,
      talks about the importance of interpersonal communication within nurses
      and their patients. They explain that good interpersonal communication
      reduces stress, promotes wellness, and overall improves their patients
      quality of life. One of the key variables they say makes a good
      interpersonal communicator is empathy. Empathy is extremely important
      because it helps someone to understand another and relate to their
      emotions. Understanding emotions is so important in communication because
      otherwise there are no connections made. Empathy is why my dad is such a
      good communicator. Another important factor in good interpersonal
      communication is actually silence, and within silence, listening. The
      journal, Human Communication and Effective Interpersonal Relationships: An
      Analysis of Client Counseling and Emotional Stability, talks about
      interpersonal communication within counselors and their patients. They
      explain that silence is actually a very important part of communication
      because the act of not speaking and listening to the person they are
      communicating shows that they find what the other person is saying is
      important to them. One important part of this is to actually listen and
      the facial expressions that they show is important to show this. Without
      listening to the other person, communication doesn't go very far. Although
      both these articles are talking about nurses and therapists with clients,
      they are still essential with daily interpersonal communications. Another
      important factor is clarity in speech. If you aren't clear about what
      you're communicating it won't go very far. Without getting the point
      across, there's not a huge point in communication. Without clear points,
      it tends to be more of a ramble or a rant. Sometimes people aren't able
      to speak clearly because their emotions are too overwhelming. It's
      important in communication to manage emotions in order to have an
      effective conversation. This is very common when people are struggling
      with anger or sadness. Sometimes their emotions make it too difficult to
      effectively communicate. Finally, nonverbal communication is very
      important in interpersonal communication. Things like facial expressions,
      posture, and gesture show your attitude towards the conversation. The
      article, Listening and Interpersonal Skills Review, explains these
      nonverbal communications and how they are apart of communication. For
      example, they explain that gestures can replace words by signaling whether
      the conversation has come to an end or not. These can show whether or not
      you're actually paying attention or if you show any interest in what they
      other person is communicating. Nonverbal communication can make or break a
      conversation. It's very important to pay attention to these because
      sometimes nonverbal communication can happen unintentionally, but they
      make a huge difference in how the other person views your position in the
      conversation. These are all factors that both my dad and boss use in order
      to properly communicate. Without using these skills, interpersonal
      communication isn't very effective.
  \end{quotation}

  \paragraph{This is a response to Julayne Kilcullen on Post ID 43148692}
    I really enjoyed reading your incredibly thorough post and I noticed a
      skills which you have listed that not many other people did which is a
      nice change-of-pace!

    Being an ``empath'' (i.e.\ someone with a high degree of empathy towards
      others) also seems to be difficult sometimes, especially when it comes
      to controlling one's emotions. Many empaths have not been doing well
      recently with the animosity in politics all over social media as well
      as the fires hurting our wildlife, not to mention the COVID-19 pandemic.
      I wonder what can be done to cope with situations like this when sometimes
      even simply leaving social media is not enough.

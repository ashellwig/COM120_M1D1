%! TEX root=../main.tex

\section{Responses}
  \subsection{Response 1}
    \begin{quotation}
      ``An understanding is the best thing in the world'' is what my dad would
        say often (Hart, 1980). He meant for you to be clear with your words so
        that others understand you. Even with him saying that to my seven
        siblings and me, I have not always used his advice, but I know this is
        the beginning to an ideal interpersonal communicator. A coworker, Carla,
        has a great way with requesting things. She is clear, direct, and asks
        for feedback as suggested by a good communicator (Stobierski, 2019). As
        an example, she may ask me if I would go get a catheter, and if I know
        where it is. The question forces me to answer yes or no to retrieving
        the catheter, and to answer if I know where it is. She is also an
        attentive listener. I’m usually always walking and doing four things at
        once or multitasking, but, when I ask her for a something, she stops,
        faces me, and acknowledges me (Headlee, 2016). Carla also uses a calm
        tone of voice. She gets loud when she laughs or when conversations are
        funny, but Carla has a calm tone. She is in control and not much can go
        array without her handling it. My brother, Jerry, uses particular body
        language when he talks to me, and I feel he is really listening to me
        when he does this. For instance, he stands straight. His hands are
        either to his side or behind him, and he faces you. His eyes are
        slightly squinted as if he is focused only on you. Both Carla and Jerry
        used an interaction model of communication where we had to talk to each
        other, verbally and non-verbally, about a request (Jones, 2013).
    \end{quotation}

    \paragraph{This is a response to Diata Hart on Post ID 43184821}
      I really enjoy the quote from your father, as it illustrates a problem
        many have in terms of communicating with each other. I find that even if
        one may \emph{not} be understood when speaking to someone, the other
        party may not make it very clear if that is the case. Not many people
        enjoy admitting to the fact they we may not know \textit{everything}.

      I, too, am envious that you were able to pick someone out of your
        workplace environment to illustrate the personification of an ideal
        interpersonal communicator. It tends to be rare that most of us get
        along with coworkers (in the working environment, not personally).
        Individuals whom are able to accept feedback and criticism with grace
        are a rarity in today's world.
